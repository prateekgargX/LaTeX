\documentclass[12pt , letterpaper]{article}

\usepackage{amsmath}
\usepackage{amsthm}
\usepackage{amssymb}

\usepackage{esint}

\usepackage{graphicx}
\usepackage{float}
%\usepackage{floatrow}

\usepackage{listings}
\usepackage{hyperref}
\usepackage{url}

\newtheorem{theorem }{Theorem}
\newtheorem{lemma}{Lemma}
%\newtheorem{corollary}[Theorem]{Corollary}
\newtheorem{proposition}{Proposition}[section]

\theoremstyle{remark}
\newtheorem*{remark}{Remark}

\begin{document}
\noindent
    \textbf{Inline Maths} embedded inside a line . like I am going to show you. 
   
    If s function $f$ is continuous at $\mathbf{x_0}$, 
    then for every $\varepsilon > 0$ $\exists $ $ \delta >0 $ 
    such that $|| x- x_o|| < \delta $ 
    implies $|f(x)-f(\mathbf{x_ab})| < \varepsilon$

    $\Theta$ and $$\theta$$

    $\mathbf{\omega}$      $\boldsymbol{\omega}$

    heheheheuhf $ p \in \mathbb{C}$ and $ n \in \mathbb{N}$ is a complex numbeer z  $z_i$. Now consider $z_i^{n^2} = Z$ . 
    We claim $ Z \in \mathbb{C}$.

    $S :=\{ x | ~x \equiv 1\pmod {9} \} ~.~ T := \{x | ~ x \equiv 1 \pmod {5} \}$ $S \subset T$

    A well known expression of $\binom{n}{k}$ is $\frac{n!}{k!(n-k)!}$ . $n!$ denotes the factorial func for non neg integers and is recursively defined as $0! := 1, n! := n\cdot(n-1)!$
    
    $$n! = \prod_{j=1}^n j.$$ A common extension of the factorial function is the Gamma function.
    For positive integere $n$, \\
    $$(n-1)! = \Gamma(n) = \int_0^{\infty}x^{n-1}e^{-x} dx$$
    
    it follows from the binomial theorem that $\sum_{k=0}^{n}\binom{n}{k} = 2^n.$ The binomial theorem is incredibly powerful, and can be used to approximate $\sqrt{1+x}$, or even $\sqrt[1+x]{71}$.
    

    $P \subseteq NP$.
    P $\subseteq$ NP.
    $\text{P} \subseteq NP$.
    However, we know that testing the positivity of the term residue($n$) is decidable in $coNO^{RP}$. 
 
    \begin{equation}
        pV= nRT
        \label{eq: ideal gas}
    \end{equation}
 
    Being a scientist as he was Vanderwaal proposed the \textit{Real} gas equation as a more accurate model.

    \begin{equation*}
        \left(p+\frac{an^2}{V^2}\right)(V-nb) = nRT
    \end{equation*}

    introducing a variable copressibility factor $Z$ , this can be expressed as 

    \begin{equation}
        pV =ZnRT \nonumber
    \end{equation}

    \begin{align*}
        \boldsymbol{\nabla} \cdot \mathbf{E} &= \frac{\rho}{\varepsilon_0} \\
        \boldsymbol{\nabla} \cdot \mathbf{B} &= 0 \\
        \boldsymbol{\nabla} \times \mathbf{E} &= -\frac{\partial \mathbf{B}}{\partial t} \\
        \boldsymbol{\nabla} \times \mathbf{B} &= \mu_0\mathbf{j} + \frac{1}{c^2}\frac{\partial \mathbf{E}}{\partial t} 
    \end{align*}

    the companion matrix $$M$$ is given by:
$$
\begin{bmatrix}
     f & 1 & 0& \dots & 0 \\
     0 & 43 & \phi & \dots & 1\\
    \vdots &\vdots &\vdots &\ddots &\vdots \\
    0&0&0&\dots&1\\ 
 \end{bmatrix}
$$

$$
\begin{pmatrix}
    f & 1 & 0& \dots & 0 \\
    0 & 43 & \phi & \dots & 1\\
   \vdots &\vdots &\vdots &\ddots &\vdots \\
   0&0&0&\dots&1\\ 
\end{pmatrix}
$$

$$
\begin{pmatrix}
    f & 1 & 0& \dots & 0 \\
    0 & 43 & \phi & \dots & 1\\
   \vdots &\vdots &\vdots &\ddots &\vdots \\
   0&0&0&\dots&1\\ 
\end{pmatrix}
$$

$$
\left\langle
\begin{matrix}
    f & 1 & 0& \dots & 0 \\
    0 & 43 & \phi & \dots & 1\\
   \vdots &\vdots &\vdots &\ddots &\vdots \\
   0&0&0&\dots&1\\ 
\end{matrix}
\right\langle
$$
    \end{document}

